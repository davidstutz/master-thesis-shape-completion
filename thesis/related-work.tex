\chapter{Related Work}
\label{ch:related-work}

In this chapter, we intend to provide a concise but structured discussion
of related work. We first consider the shape completion task itself, focusing
on data-driven and learning-based approaches that are most related to our approach.
We also discuss the use of shape priors in general, \eg for tasks including
3D reconstruction, 3D pose estimation or tracking. Finally, we give a brief
overview of 3D deep learning, specifically regarding generative shape models,
as well as the idea of amortized inference. We note, however, that we do not intend
to provide a comprehensive discussion of these areas due to time and space constraints.

\section{Shape Completion}

% PhDs:
% - Martin Oswald (http://people.inf.ethz.ch/moswald/publications/resources/Oswald-MR-PhD-Thesis.pdf)
% - Kalin Kolev
% - Christian Häne
% - Marc Pollefeys (https://www.inf.ethz.ch/personal/pomarc/pubs/PollefeysPhD.pdf)
Shape completion is closely related to 3D reconstruction as well as
surface reconstruction; however, the addressed problems
do usually not exactly match the problem of shape completion.
Therefore, we refer to recent textbooks
\cite{ForsythPonce:2012, Szeliski:2011} and surveys
\cite{MoonsVanGoolVergauwen:2010,FurukawaHernandez:2015,
BergerSilva:2014,BrecksonFisher:2005}
for a general overview of these fields.

In contrast to surface reconstruction and 3D reconstruction, shape completion
is usually performed on partial scans of individual objects.
Following \cite{SungAngstGuibas:2015}, shape completion approaches
can roughly be categorized into data-driven methods and symmetry-based
methods.
The latter detect symmetry in non-occluded parts in order to complete shapes
by completing the corresponding symmetry in occluded parts; representative works
are \cite{ThrunWegbreit:2005,PaulyGuibas:2008,ZhengChen:2010,KroemerPeters:2012}
or \cite{LawAliaga:2011}. These approaches also found application in robotics,
\eg for robot grasping \cite{KroemerPeters:2012,MartonRusuBeetz:2009}.
The data-driven case is more interesting
in relation to the proposed approaches; in early work, Pauly \etal
\cite{PaulyGuibas:2005} pose shape completion as retrieval and
alignment problem. Further work in this direction includes \eg
\cite{LiDaiGuibasNiessner:2015,NanSharf:2012,GuptaMalik:2015,DameReid:2013,
EngelmannStuecklerLeibe:2016,EngelmannLeibe:2017} where 
local shape features \cite{LiDaiGuibasNiessner:2015,NanSharf:2012},
object knowledge \cite{GuptaMalik:2015,DameReid:2013} or shape priors
\cite{EngelmannStuecklerLeibe:2016,EngelmannLeibe:2017,DameReid:2013} have found application.
Model fitting often follows a scheme similar to the
iterative closest point (ICP) algorithm \cite{BeslMcKay:1992} or is posed
as energy minimization. We also refer to \cite[Table~1]{BergerSilva:2014}.

Recently, several learning-based approaches have been proposed
\cite{RieglerGeiger:2017,FirmanBrostow:2016,SmithMeger:2017,DaiNiessner:2016,SharmaFritz:2016,
RezendeHeess:2016,FanSuGuibas:2016}, most of them using deep learning.
Shape completion is posed as supervised learning problem, \eg from single RGB(-D)
images \cite{FanSuGuibas:2016,GirdharGupta:2016}, partial observations
\cite{RieglerGeiger:2017,DaiNiessner:2016,RezendeHeess:2016,SmithMeger:2017},
or noisy shapes \cite{SharmaFritz:2016}. In comparison with data-driven approaches,
shape completion usually involves a single forward pass of the trained network.
However, these approaches can only be applied on synthetic datasets such as
ModelNet \cite{WuSongXiao:2015} or ShapeNet \cite{ChangFunkhouserGuibasSavarese:2015}
as explicit supervision is required. We also intend to leverage deep neural networks,
however, we want to learn shape completion in a weakly supervised setting
only using the given observations and knowledge about the object category.
This allows us to learn shape completion on real data, \eg on KITTI
\cite{GeigerLenzUrtasun:2012,GeigerLenzStillerUrtasun:2013}, while still
providing efficient inference through the trained network.

\subsection{Shape Priors}

As indicated above, shape priors play an important role in shape completion.
For example, learning-based approaches can be
understood as implicitly learning a shape prior from supervised training data
\cite{RieglerGeiger:2017,SmithMeger:2017,DaiNiessner:2016,SharmaFritz:2016,
RezendeHeess:2016,FanSuGuibas:2016}. Only \cite{GirdharGupta:2016} uses
the learned shape model explicitly to constrain the space of
predicted shapes. Data-driven methods may also make use of learned shape priors, \eg
using principal component analysis (PCA) \cite{EngelmannStuecklerLeibe:2016,EngelmannLeibe:2017}
or Gaussian process latent variable models (GP-LVM) \cite{DameReid:2013}.
However, shape priors have also been applied in different domains,
\eg in 3D pose estimation -- often tackled in conjunction with
tasks such as image segmentation \cite{AdrianReid:2012,DambrevilleTannenbaum:2008,
SandhuTannenbaum:2009} or scene flow \cite{MenzeHeipkeGeiger:2015}.
Shape priors have also become a standard tool in tracking, see \cite{MaSibley:2014}
or \cite{LeottaMundy:2009} for related work in this domain.
In classical 3D reconstruction, shape priors are also commonly used to resolve \eg ambiguities
\cite{GueneyGeiger:2015}, specularities \cite{DameReid:2013} or the lack of
texture \cite{BaoSavarese:2013}.
In most cases, shape priors are ``learned'' from full shapes; this is in contrast
to non-data-driven priors such as local or global smoothness priors,
planarity assumptions \etc, see \cite{BergerSilva:2014}.
We mostly follow the idea of \cite{EngelmannStuecklerLeibe:2016,EngelmannLeibe:2017,
DameReid:2013} or \cite{GirdharGupta:2016} and use a shape prior explicitly to
define and constraint the space of possible shapes.

\section{3D Deep Learning and Amortized Inference}

In computer vision, deep learning often refers to the practice of using
deep convolutional neural networks \cite{LeCunBoserDenkerHenderson:1989}
to train discriminative and more recently also generative models. 
While neural networks have been used and researched for several decades now (\eg see
\cite{Bishop:1995,Haykin:2005,Rosenblatt:1958,RumelhartHintonWilliams:1986,
HornikStinchcombeWhite:1989}), they only recently received considerable attention.
Among several milestones is the substantial performance increase on the ImageNet
classification challenge \cite{DengFeiFei:2009} in 2012 \cite{KrizhevskySutskeverHinton:2012}.
Afterwards, deep convolutional neural networks have been successfully applied
to a wide range of computer vision tasks -- more complete overviews can be found in
\cite{Schmidhuber:2014,WangRajXing:2017,Li:2017,
BengioCourvilleVincent:2012,LeCunBengioGeoffrey:2015,
GoodfellowBengioCourville:2016}.

Recently, deep generative models, \eg
\cite{VanDenOordKavukcuoglu:2016,VanDenOordGraves:2016,GoodfellowBengio:2014,
MakhzaniGoodfellow:2015,KingmaWelling:2013,LarsenWinther:2016}
and related work, have received considerable attention. Such models offer
to generate realistic images or shapes and to learn strong prior models
which are interesting for semi-, weakly- or unsupervised tasks, \eg in
\cite{KingmaMohamedRezendeWelling:2014}. In this thesis, we will use variational
auto-encoders and subsequent work \cite{ImAhnMemisevicBengio:2017,JangGuPoole:2016,MaddisonMnihTeh:2016} as they
include both an inference and a generative model. This is also motivated by
the fact that generative adversarial networks \cite{GoodfellowBengio:2014}
are known to be hard to train
\cite{MeschederGeiger:2017,NowozinTomioka:2016,SalimansGoodfellowChen:2016,
GulrajaniCourville:2017,ArjovskyBottou:2017},
%\footnote{
%  \url{https://github.com/soumith/ganhacks}
%}
a problem that we anticipated to be amplified on 3D data. Generative models
have also been applied to shape modeling, \eg
\cite{SmithMeger:2017,GirdharGupta:2016,%DaiNiessner:2016,SharmaFritz:2016,
BrockWeston:2016,WuSongXiao:2015,WuTenenbaum:2016}, allowing
shape interpolation, manipulation and generation.

\subsection{3D Deep Learning}

Most works, including \cite{WuTenenbaum:2016,SmithMeger:2017,WuSongXiao:2015,
BrockWeston:2016,SharmaFritz:2016,DaiNiessner:2016,GirdharGupta:2016,
GarciaGarciaLopez:2016,MaturanaScherer:2015}, naively
generalize convolutional neural networks to 3D data, \eg in the form of voxelized shapes.
Unfortunately, this significantly limits the used resolution; usually resolutions between
$32^3$ and $64^3$ are used. Some approaches, however, exploit the sparsity of
3D data through octrees, \eg
\cite{WangTong:2017,RieglerGeiger:2016,RieglerGeiger:2017,TatarchenkoBrox:2017},
to reduce memory and time consumption and allow higher resolutions, \eg up to
$256^3$ \cite{RieglerGeiger:2016}. Alternatively, sparse convolution schemes have
been defined to speed up training \cite{LiGuibas:2016,EngelckePosner:2017,Graham:2015}.
There is also a line of work applying convolutional neural networks directly to meshes
\cite{GuoChen:2015,BoscainiVandergheynst:2015,BrunaLeCun:2013}
or point clouds \cite{QiSuGuibas:2016a,FanSuGuibas:2016,QiYiSuGuibas:2017}. While we also
use simple 3D convolutional neural networks for the presented experiments,
which we also conduct on a resolution of $32^3$, the above references provide
interesting possibilities for future work.

\subsection{Amortized Inference}

% https://arxiv.org/pdf/1611.01722.pdf
% http://gershmanlab.webfactional.com/pubs/GershmanGoodman14.pdf
% https://arxiv.org/pdf/1505.05770.pdf
To the best of our knowledge, the notion of amortized inference
was first introduced in \cite{GershamGoodman:2014}, however, has been used repeatedly
in recent work
\cite{KingmaWelling:2013,RezendeMohamed:2015,WangLiu:2016,
RezendeMohamedWierstra:2014,RitchieGoodman:2016}. While a clear, coherent definition
is still missing, amortized inference generally describes the idea of learning
inference, \eg in the form of ``learning to sample'' \cite{WangLiu:2016}
or amortized, \ie learned, variational inference \cite{KingmaWelling:2013,
RezendeMohamed:2015}. The idea includes that drawn samples -- or past inferences -- 
are not considered independently; instead, inference can profit from insights
obtained from past inferences. In our case, we learn, \ie amortize, the 
maximum likelihood problem for shape completion.
