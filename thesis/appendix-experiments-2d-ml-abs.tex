\begin{figure}
  \centering
  \begin{tikzpicture}
    \begin{axis}[
        ybar stacked,
        % https://tex.stackexchange.com/questions/119887/remove-the-scientific-notation-which-is-unreasonable
        yticklabel style={
          /pgf/number format/fixed,
          /pgf/number format/precision=5
        },
        scaled y ticks=false,
        %enlargelimits=0.15,
        legend style={
          at={(0.75,0.95)},
          anchor=north west,
        },
        % https://tex.stackexchange.com/questions/48620/pgfplots-alignment-and-size-of-math-in-legend
        legend cell align=left,
        xtick={
          1, 2,
          3, 4, 5,
          6, 7, 8,
          9, 10, 11,
          12, 13, 14, 15,
          16, 17, 18, 19
        },
        xticklabels={
          \PPCA\\$Q = 5$, \VAE\\$Q = 5$,
          \ML \PPCA\\\easy, \ML \PPCA\\\moderate, \ML \PPCA\\\hard,
          \AML \PPCA\\\easy, \AML \PPCA\\\moderate, \AML \PPCA\\\hard,
          \ML \VAE\\\easy, \ML \VAE\\\moderate, \ML \VAE\\\hard,
          \AML \VAE\\\easy, \AML \VAE\\\moderate, \AML \VAE\\\hard, \AML \VAE\\\hard *,
          \EVAE\\\easy, \EVAE\\\moderate, \EVAE\\\hard, \EVAE\\\hard *,
        },
        x tick label style={text width=1.5cm,align=right},
        ymin=0,
        ymax=0.16,
        width=15.5cm,
        height=4cm,
        enlarge x limits=0.06,
        % https://tex.stackexchange.com/questions/271027/pgfplots-how-to-rotate-extra-x-tick-labels
        x tick label style={
          rotate=90,
          anchor=east,
        },
      ]
      
      % AbsThr
      \addplot coordinates {
        (1, 0.0449326171875)
        (2, 0.00318352)
        %
        (3, 0.0990908203125)
        (4, 0.0997822265625)
        (5, 0.0992470703125)
        %
        (6, 0.07406222)
        (7, 0.07938213)
        (8, 0.09560468)
        % 
        (9, 0.09885333)
        (10, 0.09952964)
        (11, 0.10259025)
        %
        (12, 0.03621944)
        (13, 0.05189751)
        (14, 0.10733886)
        (15, 0.07285957)
        %
        (16, 0.0369839)
        (17, 0.05154616)
        (18, 0.10695919)
        (19, 0.07254225)
      };
      \addlegendentry{\AbsThr}
      
      % Abs
      \addplot coordinates {
        (1, 0.05237) % 0.0973753175757)
        (2, 0.00205) % 0.00515391)
        %
        (3, 0.03837) % 0.137378104094)
        (4, 0.0382) % 0.13795854836)
        (5, 0.03816) % 0.137437428765)
        %
        (6, 0.0296) % 0.10367236)
        (7, 0.02646) % 0.10546739)
        (8, 0.01173) % 0.10733372)
        % 
        (9, 0.02035) % 0.11928853)
        (10, 0.0238) % 0.12332883)
        (11, 0.0191) % 0.12160376)
        %
        (12, 0.0032) % 0.03942102)
        (13, 0.0013) % 0.05316359)
        (14, 0.004) % 0.11131828)
        (15, 0.0016) % 0.07475822)
        %
        (16, 0.0018) % 0.03876815)
        (17, 0.0021) % 0.05361402)
        (18, 0.005) % 0.11190726)
        (19, 0.00245) % 0.07497483)
      };
      % Abs
     % \addplot coordinates {
     %   (1, 0.0973753175757)
     %   (2, 0.00515391)
        %
     %   (3, 0.137378104094)
     %   (4, 0.13795854836)
     %   (5, 0.137437428765)
        %
     %   (6, 0.10367236)
     %   (7, 0.10546739)
     %   (8, 0.10733372)
        % 
     %   (9, 0.11928853)
     %   (10, 0.12332883)
     %   (11, 0.12160376)
        %
     %   (12, 0.03942102)
     %   (13, 0.05189751)
     %   (14, 0.10733886)
     %   (15, 0.07285957)
        %
     %   (16, 0.0369839)
     %   (17, 0.05361402)
     %   (18, 0.11190726)
     %   (19, 0.07497483)
     % };
      \addlegendentry{\Abs}
    \end{axis}
  \end{tikzpicture}
  % TODO short caption
  \caption{Comparison of all proposed approaches, \ML, \AML and \EVAE
  on the 2D dataset for \easy, \moderate and \hard difficulties
  with respect to absolute error \Abs and thresholded absolute error \AbsThr. We
  only consider occupancy, for signed distance functions as shape representation,
  we refer to Figure \ref{fig:experiments-2d-aml}. For \AML
  and \EVAE we additionally compare the use of the weighted maximum likelihood
  loss, marked by *. Refer to the text for details.}
  \label{fig:experiments-2d-ml}
\end{figure}
